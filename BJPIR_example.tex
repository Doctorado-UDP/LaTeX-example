\documentclass[12pt,halfline,a4paper]{ouparticle}
\usepackage[english]{babel}
\usepackage[colorlinks=true,citecolor=blue,linkcolor=blue,urlcolor=blue,hyperfootnotes=false]{hyperref}
\usepackage{natbib}
\usepackage[utf8]{inputenc}
\usepackage{booktabs}
\usepackage{multirow}
\usepackage[dvipsnames]{xcolor}
\usepackage{marvosym}
\usepackage{fontawesome}
\usepackage[bottom]{footmisc}

\usepackage{silence}
%% \WarningFilter{latex}{Text page 6 contains only floats}

\usepackage[T1]{fontenc}
%% \usepackage{tipa}

\begin{document}

\title{Ministerial Stability During Presidential Approval Crises: The Moderating Effect of Ministers’ Attributes on Dismissals in Brazil and Chile\thanks{This research has been funded by the Chilean National Agency for Research and Development (ANID/PFCHA/72200340). Acknowledgements to a number of scholars and colleagues can be found at the end of the document. The author declares no potential conflict of interest with respect to this research.}}

\author{%
\name{Bastián González-Bustamante\thanks{{\faMapMarker} St Hilda's College, Cowley Place, Oxford {\scriptsize OX4 1DY}, {\large \Letter} \href{mailto:bastian.gonzalezbustamante@politics.ox.ac.uk}{bastian.gonzalezbustamante@politics.ox.ac.uk}, {\large \Letter} \href{mailto:bastian.gonzalez.b@usach.cl}{bastian.gonzalez.b@usach.cl}, {\faHome} \href{https://bgonzalezbustamante.com/}{https://bgonzalezbustamante.com}, {\scriptsize ORCID iD} \href{https://orcid.org/0000-0003-1510-6820}{https://orcid.org/0000-0003-1510-6820}.}}
\address{University of Oxford \\ Universidad de Santiago de Chile}
%% \email{bastian.gonzalezbustamante@politics.ox.ac.uk}
%% \and
%% \name{Co-Author}
%% \address{Affiliation}
%% \email{Email}
}

\abstract{This paper analyses the effect of ministers’ exposure to periods of low presidential approval in Brazil and Chile between 1990 and 2014. Approval is explored with quarterly estimates using a dyad-ratios algorithm and merged into a time-dependent cabinet data set to evaluate individual ministerial terminations ($N$ = 4,245). The empirical strategy combines time-varying exposure Cox regressions with observational data and propensity score and matching to estimate the effect of low approval on ministerial survival and perform a moderation analysis with three profiles associated with presidential strategies: (i) nonpartisan ministers to limit agency loss and moral hazard; (ii) economists as ministers to optimise cabinet performance and send positive signals to the electorate; and (iii) party leaders as ministers to optimise legislative support. The main findings show that risk increases by 135.1\% in periods of low approval. In addition, approximately only one in five nonpartisan ministers is removed compared to party members.}

%% \date{\today}
\date{{\normalsize October 2, 2022} \\ {\footnotesize This article has been accepted in the {\itshape British Journal of Politics and International Relations}}}

\keywords{ Brazil; cabinets; Chile; ministerial turnover; presidential approval; propensity score; survival analysis}

\maketitle

\section{Introduction}
\label{sec1}

In May 2010, just a month after the start of Sebasti\'an Pi\~nera’s first presidential term, Jaime Ma\~nalich, the recently appointed health minister, was summoned to testify about an allegedly falsified alcohol test performed seven months earlier on Pi\~nera’s brother after a traffic accident. The test took place at a private clinic where Ma\~nalich was then general manager and Pi\~nera was a shareholder. Regardless of public pressure, Ma\~nalich remained health minister throughout the four-year administration. A very different situation occurred in June 2020, when Pi\~nera made the fourth cabinet change of his second term (2018-2022). Despite persistent questioning of the government’s response to the coronavirus pandemic, the change consisted only in a reshuffle of the same ministers, designed partly to protect questioned Health Minister Jaime Ma\~nalich. The protagonists are the same as in 2010 but the outcome was entirely different: only nine days after the reshuffle, Ma\~nalich resigned in the face of criticism of the inefficiency of the dynamic quarantines implemented by the government and a mismatch between the statistics reported by the Health Ministry and the World Health Organization (WHO). Thus, while the President was again willing to protect Ma\~nalich, the pressure was unsustainable.

This case is an excellent example of how a president can protect ministers from calls for their resignation and the effect of scandals. In this particular example, Pi\~nera reappointed Ma\~nalich as a minister in his second term but, this time, was unable to protect him. Moreover, this is a case of a nonpartisan minister close to the president and illustrates how a president can limit moral hazard and agency loss by appointing and protecting independent ministers close to his entourage or inner circle and without partisan loyalties.

While ministers’ attributes and trajectories are a recurrent topic of study from different approaches, theoretical arguments tend to result in a conceptual heterogeneity that complicates empirical research \citep{Camerlo2018}. This heterogeneity translates into a variety of concepts for identifying profiles that are complex to measure empirically, posing a methodological challenge for testing how ministers’ profiles offer advantages in the face of unexpected events or shocks that impact governments’ stability. In this context, we consider periods of low presidential approval, a situation common to almost every administration that should impact actors’ incentives and strategies. 

Thus, our main question is: How can a minister’s attributes prevent his exit from the cabinet during periods of low presidential approval? Answering this question allows us to offer an empirically approachable conceptualisation of ministerial profiles linked to presidential strategies in contexts of approval crises and, in this way, make a theoretical contribution to bridging the gap in the general conceptualisation of profiles. In addition, we propose a specific procedure for correctly estimating effects and bias using the survival approach, which constitutes a substantial, novel methodological contribution. This contribution is relevant since straightforward modelling of the relationship between low approval and cabinet turnover may be biased and exposed to endogeneity because presidents or prime ministers tend to reshuffle their cabinets at times of low popularity \citep{Kam2005, MartinezGallardo2014}.

We will refer to two country-specific cases: Brazil and Chile. Although both are cases of multi-party coalitions, political alliances in Chile have been remarkably stable in recent decades and, for instance, Chile has more robust party organisation capabilities \citep{Martinez2021}. In addition, both cases have shown a higher degree of technical control over economic policies in recent decades compared to other countries in the region, such as Argentina, where technocracy has gradually declined \citep{Dargent2015}. Finally, as evidenced below, both countries have experienced periods of low presidential approval and comparable levels of ministerial survival in the medium to long term.

In the next section, we present the theory and empirical expectations, reflecting first on stochastic events, low approval and ministerial recruitment before going on to connect recruitment and political careers with different ministers’ profiles and principal-agent, signalling and coalition theories. We then present our empirical strategy with the encoding of the time-dependent data set, detailed information on the regressions with observational data and the application of propensity score and matching along with placebo tests and robustness checks. The main results are then presented with the estimation of the effect of exposure to low approval on the exit of cabinet ministers, moderation analyses and additional tests. Finally, the main findings are summarised and briefly discussed.

\section{Ministerial Turnover and Ministers’ Profiles}
\label{sec2}

\subsection{Stochastic Events, Low Approval and Ministerial Recruitment and Careers}
\label{sec2.1}

The concept of crisis is associated with stochastic events in the framework of the event-based approach in government survival literature, especially in the case of parliamentary systems \citep{Browne1986, Fortunato2018, King1990, Warwick1994}. This concept accounts for unforeseen events or shocks that affect the stability of governments. The magnitude of their effect depends on different factors, such as the complexity of the bargaining environment and the party system’s characteristics. More complex environments render governments more sensitive to these events \citep{Chiba2015, Laver1996, Laver1990}. Such events can be critical and generate a renegotiation of the balance of power as it stood at the beginning of a presidential term and may even lead to a reordering of parties and coalitions. Both situations can, in turn, trigger a cabinet reshuffle \citep{MartinezGallardo2012}.

Events that typically affect cabinet stability include protests, economic crises, media scandals, corruption cases, low approval and natural disasters of different sorts \citep{Camerlo2015a, MartinezGallardo2014}. It is to be expected that different events will have different effects on the actors’ behaviour depending on the specific context. However, as \cite{Berlinski2010} indicate, evaluating all possible random shocks is empirically complex and we, therefore, opted to focus on periods of low presidential approval. 

\section{Empirical Strategy}
\label{sec3}

\subsection{Time-Dependent Data Encoding}
\label{sec3.1}

\subsection{Time-Varying Exposure Cox Regressions}
\label{sec3.2}

\bibliographystyle{apalike}
\bibliography{refs/BIB-Ministers-Approval}
\addcontentsline{toc}{section}{References}

\end{document}