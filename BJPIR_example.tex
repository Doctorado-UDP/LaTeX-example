\documentclass[12pt,halfline,a4paper]{ouparticle}
\usepackage[english]{babel}
\usepackage[colorlinks=true,citecolor=blue,linkcolor=blue,urlcolor=blue,hyperfootnotes=false]{hyperref}
\usepackage{natbib}
\usepackage[utf8]{inputenc}
\usepackage{booktabs}
\usepackage{multirow}
\usepackage[dvipsnames]{xcolor}
\usepackage{marvosym}
\usepackage{fontawesome}
\usepackage[bottom]{footmisc}

\usepackage{silence}
\WarningFilter{latex}{Text page 6 contains only floats}

\begin{document}

\title{Ministerial Stability During Presidential Approval Crises: The Moderating Effect of Ministers’ Attributes on Dismissals in Brazil and Chile\thanks{This research has been funded by the Chilean National Agency for Research and Development (ANID/PFCHA/72200340). Acknowledgements to a number of scholars and colleagues can be found at the end of the document. The author declares no potential conflict of interest with respect to this research.}}

\author{%
\name{Bastián González-Bustamante\thanks{{\faMapMarker} St Hilda's College, Cowley Place, Oxford {\scriptsize OX4 1DY}, {\large \Letter} \href{mailto:bastian.gonzalezbustamante@politics.ox.ac.uk}{bastian.gonzalezbustamante@politics.ox.ac.uk}, {\large \Letter} \href{mailto:bastian.gonzalez.b@usach.cl}{bastian.gonzalez.b@usach.cl}, {\faHome} \href{https://bgonzalezbustamante.com/}{https://bgonzalezbustamante.com}, {\scriptsize ORCID iD} \href{https://orcid.org/0000-0003-1510-6820}{https://orcid.org/0000-0003-1510-6820}.}}
\address{University of Oxford \\ Universidad de Santiago de Chile}
%% \email{bastian.gonzalezbustamante@politics.ox.ac.uk}
%% \and
%% \name{Co-Author}
%% \address{Affiliation}
%% \email{Email}
}

\abstract{This paper analyses the effect of ministers’ exposure to periods of low presidential approval in Brazil and Chile between 1990 and 2014. Approval is explored with quarterly estimates using a dyad-ratios algorithm and merged into a time-dependent cabinet data set to evaluate individual ministerial terminations ($N$ = 4,245). The empirical strategy combines time-varying exposure Cox regressions with observational data and propensity score and matching to estimate the effect of low approval on ministerial survival and perform a moderation analysis with three profiles associated with presidential strategies: (i) nonpartisan ministers to limit agency loss and moral hazard; (ii) economists as ministers to optimise cabinet performance and send positive signals to the electorate; and (iii) party leaders as ministers to optimise legislative support. The main findings show that risk increases by 135.1\% in periods of low approval. In addition, approximately only one in five nonpartisan ministers is removed compared to party members.}

%% \date{\today}
\date{{\normalsize October 2, 2022} \\ {\footnotesize This article has been accepted in the {\itshape British Journal of Politics and International Relations}}}

\keywords{ Brazil; cabinets; Chile; ministerial turnover; presidential approval; propensity score; survival analysis}

\maketitle

\section{Introduction}
\label{sec1}

\section{Ministerial Turnover and Ministers’ Profiles}
\label{sec2}

\subsection{Stochastic Events, Low Approval and Ministerial Recruitment and Careers}
\label{sec2.1}

\section{Empirical Strategy}
\label{sec3}

\subsection{Time-Dependent Data Encoding}
\label{sec3.1}

\subsection{Time-Varying Exposure Cox Regressions}
\label{sec3.2}

%% \bibliographystyle{apalike}
%% \bibliography{refs/}
%% \addcontentsline{toc}{section}{References}

\end{document}